\documentclass{article}
\usepackage{ctex}
\usepackage[a4paper]{geometry}
\usepackage{amsthm,amsmath,amssymb}
\usepackage{graphicx}
\usepackage{gensymb}
\usepackage{url}
\usepackage{hyperref}
\usepackage{enumitem}
\usepackage{ulem}
%\usepackage{soul}
\usepackage{color}

\begin{document}
\section{哈哈}
可视化中 
\textcolor{red}{背景颜色、透明度、阴影}的使用
\section{哈哈1}
\emph{异常处理中“Throws early, catch late”原则} \\
\cite{exceptionhandling}
假设一个函数在接收错误的参数时出现bug,按照这个原则,当程序出现问题的时候你就能立刻知道问题是出在函数的调用上。
如果直到用到整个参数的时候才进行处理,那么要想找到问题所在就需要按着程序的执行反向推导才可以。
所以尽早的抛出异常能够让你更容易的找到问题的所在。
异常的处理一般放在high-level中,因为lower-level可能不知道什么是合适的处理方法。事实上,
可能有多种合适的处理方法。拿打开文件为例,如果打开一个不存在的配置文件,
合理的处理方法是忽略这个异常转而使用默认的配置文件; 如果打开的是一个可执行文件但是缺少一些东西,
那么往往只能关闭程序的执行。

另外,在不同的level中需要进行合理的异常包装,转换成符合当前执行环境的异常类型。
\begin{quotation}
  "Wrapping the exceptions in the right types is a purely orthogonal
  concern."
\end{quotation}
还有再有就是异常的处理是为了让程序从错误中恢复,如果在该阶段进行处理不能达到这个目的,那么就不应该处理。

\bibliographystyle{plain}
\bibliography{ref_in_mind}
\end{document}


